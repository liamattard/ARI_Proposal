With over 20,000 FDA approved medications that are prescription-only,
doctors may face a challenge when giving out medicine to a specific
patient. Unfortunately, the FDA receives more than 100,000 declarations
of medication errors each year in the United States alone \cite{FDA2021}.
As a result, prescribing medicine to patients remains a challenge, and
modern solutions are moving from the all-purpose medicine concept to
personalised health care \cite{Bhoi2021}.

Modern hospitals use Electronic Health Records (EHR) to keep track of
everything and control the complexity of data. EHRs are a collection of
clinical information gathered from health care patients. The mass
adoption of such systems delivers a large amount of data compiled on a
patient's visits, such as demographics, diagnosed conditions, medical
prescriptions, procedures and any health-related history \cite{Kim2019}.

This data is being used for machine learning systems to improve and
automate clinical care practices, for example, early disease detection
and identifying patients at high risk of severe conditions
\cite{10.2307/20720782, Juhn2019}.

A system that recommends a list of medicine based on a patient's current
state will serve as an essential decision-support tool for medical
experts to assist with drug prescriptions and could help prevent
prescription errors \cite{Jamshidi2018}. 

Recommender Systems (RS) are techniques that derive patterns and suggest
items to a user\cite{Ricci2011}. Combining RSs with an EHR dataset develops 
a system that recommends personalised medicine by learning from
the prescribed drugs and the patient information. Existing medicine
recommender systems make use of user reviews or past
diagnosis \cite{Bhoi2021, Rao2020}.

However, medicine recommender systems come with both opportunities and
challenges. The most difficult challenge is perhaps the lack of available
health-related data online. Ibrahim et al. \cite{Ferner2000} stated that
health data poverty stops individuals from implementing data-driven
technologies that benefit healthcare. Another data-related challenge is
that most health data is anonymised for privacy concerns which is a good
thing; however, it presents a challenge for systems to learn. Unlike data
from an e-commerce website, medical data in an EHR is implicit, meaning
there is no rating about each user-item interaction.

%This study will try to answer
%the following research question: \begin{center} \textit{ Can a
        %recommender system use techniques such as collaborative filtering
%to suggest a list of medication personalised to a patient? } \end{center}

