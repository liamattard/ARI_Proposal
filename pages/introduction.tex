With over 20,000 prescription-only FDA approved medications,
doctors may face a challenge when prescribing medicine to a
specific patient. Unfortunately, the FDA receives more than
100,000 declarations of medication errors each year in the United
States alone \cite{FDA2021}. Modern hospitals use Electronic
Health Records (EHR) to keep track of everything and deal with
this complexity \cite{Kim2019}. 

EHRs are a collection of clinical information gathered
from health care patients. The mass adoption of such
systems deliver a large amount of data compiled on a
patient's demographics, diagnosed conditions, medical
prescriptions, procedures and any health-related history
\cite{Kim2019}.

%\subsection{Different AI concepts being brought together }

This data provides opportunities for machine learning systems to improve and
automate clinical care practices, for example, early disease detection and
identifying patients at high risk of severe conditions
\cite{10.2307/20720782, Juhn2019}.


A system that suggests a list of medicine based on a patient's current state will serve as an essential decision-support tool for medical experts 
to assist with patient prescriptions. Recommender Systems (RS) are
techniques to derive patterns and deal with complex drugs and user
information to recommend personalised results.
%TODO: Cite

%\subsection{Research Question}

In this study, we will try to answer the following research question:
\begin{center}
\textit{Can a Recommender System predict and suggest a list of personalised medication to
a patient using information from an EHR.}
\end{center}


%This data provides opportunities for machine learning systems,
%such as Recommender Systems (RS) to automate a particular
%hospital procedures. For example, A system that suggests a list
%of medicine based on a patient's current tate, will serve as an
%important decision-support tool for medical experts to assist
%with patient prescriptions \cite{Bhoi2021}.

