With over 20,000 prescription-only FDA approved medications, doctors may face a
challenge when prescribing medicine to a specific patient. Unfortunately, the
FDA receives more than 100,000 declarations of medication errors each year in
the United States alone \cite{FDA2021}. Modern hospitals use Electronic Health
Records (EHR) to keep track of everything and deal with this complexity
\cite{Kim2019}. 

EHRs are a collection of clinical information gathered from health care
patients. The mass adoption of such systems delivers a large amount of data
compiled on a patient's visits, such as demographics, diagnosed conditions,
medical prescriptions, procedures and any health-related history
\cite{Kim2019}.

This data provides use cases for machine learning systems to improve and
automate clinical care practices, for example, early disease detection and
identifying patients at high risk of severe conditions \cite{10.2307/20720782,
Juhn2019}.

A system that recommends a list of medicine based on a
patient's current state will serve as an essential
decision-support tool for medical experts to assist with
drug prescriptions. Recommender Systems (RS) are
techniques that derive patterns and suggest items to a
user\cite{Ricci2011}. Providing RSs with EHR data of drugs and patient
information could result in a system that recommends
personalised medical results. Existing medicine recommender systems make use of user
reviews or past user diagnosis \cite{Bhoi2021, Rao2020}.
This study will try
to answer the following research question:
\begin{center} \textit{
Can a recommender system use techniques such as
collaborative filtering to suggest a list of
medication personalised to a patient by using
information from an EHR such as demographics,
diagnosis, and physiologic data?
}
\end{center}
