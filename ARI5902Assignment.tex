\documentclass[conference]{IEEEtran}
\usepackage{cite}
\usepackage{amsmath,amssymb,amsfonts}
\usepackage{algorithmic}
\usepackage{graphicx}
\usepackage{textcomp}
\usepackage{xcolor}
\def\BibTeX{{\rm B\kern-.05em{\sc i\kern-.025em b}\kern-.08em
T\kern-.1667em\lower.7ex\hbox{E}\kern-.125emX}}
\begin{document}

\title{Medicine Recommender System}

\author{\IEEEauthorblockN{Liam Attard}
	\date{January 2022}
}

\maketitle

\begin{abstract}
    \bigskip
\bigskip
\bigskip
\bigskip
\bigskip
\bigskip
\bigskip
\bigskip
\bigskip
\bigskip
\bigskip
\bigskip
\bigskip
\bigskip
\bigskip
\bigskip
\bigskip

\end{abstract}

\begin{IEEEkeywords}
    Recommender Systems, Health Care AI, Collaborative Filtering,
    Electronic Health Records.
\end{IEEEkeywords}

\section{Introduction}
    With over 20,000 prescription-only FDA approved
medications \cite{Sun}, doctors may face a challenge when
prescribing medicine to a specific patient.
Unfortunately, the FDA receives more than 100,000
declarations of medication errors each year in the
United States alone. Modern hospitals use Electronic
Health Records (EHR) to keep track of everything and
deal with this complexity. 

EHR are a collection of clinical information gathered
from health care patients. The mass adoption of such
systems delivers a large amount of data compiled on a
patient's diagnosed conditions, medical prescriptions,
procedures and any health-related history.

This data provides opportunities for machine learning
systems, such as Recommender Systems (RS), to aid
medical experts in prescribing personalised medication
for each patient.


\subsection{Different AI concepts being brought together }

\subsection{Research Question}


\section{Aim and Objectives}
    \label{AimObjectives}
The new era of precision medicine brings forward systems that can work
hand in hand with clinicians by personalising the patient's results and
preventing medication errors. This project aims to build a health RS that
uses a person's current state and past medical history (demographics,
diagnosis, and physiologic data) to recommend a set of personalised
medications from an EHR. We will try to answer the following research
question:

\begin{center}
    \textit{
        Can personalised recommendations suggest suitable
        medicine for patients?
    }
\end{center}

%\subsection{Aim}

We form the following objectives to achieve our aim:

\begin{itemize}

    \item Address cold start issues when introducing new patients.

     \item Find the best personalisation/recommendation
         model suited for all possible scenarios.

    \item Make sure that any pair of recommended medicine does not
        cause any adverse side effects with each other.

    \item 
        Determine essential features in the EHR that hold the most value
        when predicting medicine to patients.

\end{itemize}


In the upcoming sections, we will discuss the existing
literature while giving an overview of the RS research
area in section \ref{Background}. We will then propose our solution to
the problem in section \ref{Proposed} and provide ways of evaluating
our system in section \ref{Testing}. 



\section{Related Work}
    This section briefly explains the medicine RS research area, lists similar
systems, and explains what technologies their researchers used to build
them. When available, we also try to look into their solutions for the
objectives listed in section \ref{AimObjectives}. We will split the
research into two categories, described in the following subsections.

\subsection{Ontology and Rule-Based Systems}

These systems typically rely on a set of rules, referred to as a knowledge
base, that mimic the logic of a medical expert \cite{Stark2019}.

GalenOWL \cite{Doulaverakis2012} recommends drugs to patients based on the
user's diseases, allergies and drug interactions. The system stores the
rules in a Resource Description Framework graph and uses SPARQL to query
the knowledge space. The results were reviewed and evaluated by a medical
expert, and it was stated in future work that there should be an expansion
of semantic rules. 

Doulaverakis et al. created the Panacea \cite{Doulaverakis2014} system based on
GalenOwl's approach. They use SKOS vocabulary for the rules. Results show
better query performance over GalenOwl and accommodate a far greater knowledge
base.


Lakkaraju et al.\cite{pmlr-v54-lakkaraju17a} model their RS using a decision list
as a Markov Decision Process (MDP) and Upper Confidence Bound Trees to
prune the search space efficiently. Their system maps between demographics
and treatments and try to maximise outcomes, minimise treatment costs and
expresses this into an interpretable tree-based model. They evaluated
their system by using criminal justice and health care domains. 


Not all medical RS focus on recommending medicine to all types of
patients; some decide to focus on a specific group of people. For example,
IRS-T2D \cite{Mahmoud2016} is a system built to personalise medicine for
patients with type 2 diabetes. The solution uses Semantic Web Rule Language
from HbA1c  tests, anti-diabetics, and dose selection restrictions. To evaluate
their system, they used a dataset made up of 30 patients and achieved a very
high precision rate. 

Hamed et al. \cite{Hamed2012} make a different approach. They analyse a set of
500,00 tweets to recommend medication. They start by checking the condition of
a person, send them a questionnaire to get more info and apply a C4.5 decision
tree algorithm to predict the condition of the user. Based on the result, the
algorithm can derive the correct medical product. 

While rule-based approaches offer reasonable solutions, they also have
advantages and disadvantages. Since everything is based on rules, they are
more prone to cold start issues \cite{PradeepKumarSingh2021}. However, They are
not easy to scale, and it is challenging to add rules to an ample domain space
without introducing conflict \cite{Bhoi2021}.

\subsection{
Content-Based, Collaborative Filtering and machine learning-based
}

There are two main ways of approaches to building Machine-Learning based
RS. The first method is using Collaborative Filtering (CF). Using this
approach, an RS recommends items to a user based on the choices of similar
users. The algorithm measures user similarity from features like
demographics, diagnosis or prescriptions. Another approach is using
content-based filtering (CF), which uses item similarity instead. For
example, if a user was given medicine A with similar features to medicine
B, that is recommended. 

CF is used by CADRE \cite{Zhang2015}, trained on the Walgreens dataset, which unfortunately has been removed. CADRE is a cloud-based RS that performs top-N recommendations in three steps. The data pre-processing
part is for clustering and cleaning the data, and after CF, tensor
decomposition addresses the sparsity problem of CF. Unfortunately, this
system achieved low results; however, not many features were used to train
the model, and the dataset contains only around 6000 users. They state
that future work could use user demographics to increase F1, accuracy and
recall scores.

The Mimic dataset is a publicly available EHR dataset that will be
described in detail later in section X. Several RS use such datasets like
LEAP, PREMIER and Wang et al.'s system. 


Zhang et al. \cite{LEAP} created LEAP , an end-to-end learning algorithm that uses a
recurrent decoder and content-based attention for treatment recommendation from
disease-drug mapping. Leap used reinforcement learning to fine-tune the model
and achieved a 10\% increase performance overall baselines. Bhoi et al. 
cite{Bhoi2021} categorised this approach as an instance-based system that
only recommends medicine based on current visits and does not consider past
visits. 

\begin{figure}
    \includegraphics[width=8.5cm, height = 8cm]{PREMIER.png}
    \caption{PREMIER's two Stage Recommender System.}
    \label{one}
\end{figure}

Recurrent Neural Networks were popular architectural choices for mimic-based RS
mainly because of their memory ability which is vital for training on user
visits \cite{Wang}. PREMIER \cite{Bhoi2021} is a two-stage attention-based
RS. Figure X shows the architecture with information about each stage. In
the first stage, the system uses past diagnoses and procedures and embeds
this information into the RNN. In the second stage, they combine the
second dataset of drug interactions to ensure that each recommendation is
safe for the user. The system also justifies the recommendations by
splitting them into two parts; one for the diagnosis and one for the
procedures, and uses a weight feature to calculate the importance. As a
result, PREMIER outperforms state-of-the-art medication recommendation
systems while achieving the best tradeoff between accuracy and drug-drug
interaction.  

Figrue \ref{one} Finally, Wang et al. \cite{Wang} proposed a solution that uses Supervised Reinforcement
Learning with RNNs on the Mimic Dataset. They contain an off-policy
actor-critic architecture to discover unique optimal personalised
treatments and evaluated that their system can decrease the estimated
mortality in hospitals by up to 4.4\%.



%Health RS play an essential role in research such as food and diet, physical activity and medicine recommendation. Medicine RS solutions classify into either ontology and rule-based approaches or, more recently, machine-learning approaches. Both collaborative filtering (CF) and content-based (CB) techniques have been used for the medicine RS to work. Generally, researchers use EHR gathered from hospitals or medicine ratings and reviews from websites to train their models or algorithms.
    

%The MIMIC (Medical Information Mart for Intensive Care) is a publicly available EHR dataset under the registration of the university of MIT containing data from patients admitted to the critical care units of the Beth Israel Deaconess Medical Center. There are four versions of this database. The third version contains twenty-six tables, over forty thousand patient records, and around sixty thousand patient visits. Each patient is de-identified and anonymised and contains information about the prescriptions, diagnosis, medications, demographics, doctor's notes and more. We want to use this dataset to train our RS model and intend to combine this dataset with a Drug Interactions dataset such as Drugs.com's system to ensure safe recommendations. Furthermore, this dataset provides the opportunity to build a hybrid recommender system that uses the patient data for CF techniques and doctor's notes for semantic analysis and CB techniques. 

%RS typically uses user rating clicks or explicit feedback to build and grade their system. Since an EHR does not contain such data, we intend to incorporate implicit CF techniques or a hybrid RS to improve existing systems. 


%By splitting the dataset into training and testing, we could evaluate our RS model and compare it with existing solutions. Moreover, The existing Drug Interactions evaluation techniques will aid in evaluating the safety of our RS. 

\section{Proposed Idea}
    \subsection{Dataset}
    Health RS plays an essential role in food, diet, and physical activity
recommendation, and RS research is growing fast, especially in the e-commerce
sector \cite{Tran2021b}. However, the RNN prediction solutions
like Wang et al. \cite{Wang} and PREMIER \cite{Bhoi2021} only incorporate past diagnoses and past
procedures. Our proposed solution for this study is to use the vast amount of
data in EHR like MIMIC in the RS.  The following describes our proposed ways of
tackling the objectives.


\subsubsection{
    Tackling cold start issues for new patients
}

The more features the system contains, the more it can handle cold start issues
for new patients with little data. For example, if the system is trained on the
age, diagnoses and procedures and a new user only has the first two, the system
should be capable of recommending the right medicine. 

\subsubsection{
Finding the best Model
}
After splitting the dataset into a training set and a testing set, we would
like to find the best model that achieves the highest scores.

\subsubsection{
    Preventing adverse side effects from recommendations
}
We can aggregate the EHR dataset with a drug interactions dataset containing
information about a specific medication and its effects and conflicts with
other drugs. This combination ensures that any recommended medicine does not
cause harm or damaging side effects to any patients. 

\subsubsection{
    Selecting the best features from the dataset
}
Choosing the best features from the EHR dataset helps us build an efficient
model. Our goal is to determine which data hold value for increasing accurate
predictions.


\subsection{
The Dataset 
}
The MIMIC (Medical Information Mart for Intensive Care) is a publicly
available EHR dataset under the registration of the university of MIT
containing data from patients admitted to the critical care units of the Beth
Israel Deaconess Medical Center \cite{Johnson2016}.

There are four versions of this database, and after applying, we have been
given access to MIMIC III and MIMIC IV. The third version contains twenty-six
tables, over 40,000 patient records (excluding people under 16), and 53,423
visits between 2001 and 2012. Each patient is de-identified and anonymised and
contains medical intake, chart events, ICU date stays, demographics, doctor's
notes and more. On average, each patient has around 2.68 visits, and table \ref{age}
shows some statistics about the dataset, and figure \ref{statistics} shows the age
distribution of the patients. People over 90 are grouped. Each patient is
identified with a patient ID, used throughout the tables. Diagnosis, Procedures
are encoded using international standards such as ICD-9 and ATC classification .

\begin{figure}[h]
    \includegraphics[width=8.5cm, height = 6cm]{ageDistribution.png}
    \caption{Age Distribution of MIMIC III}
    \label{age}
\end{figure}


\begin{table}[h]

    \caption{MIMIC III Statistics.}
    \label{statistics}
\begin{center}
\begin{tabular}{ | c | c | }
    \hline
 Number of patients     & 46520 \\ 
    \hline
 Number of Diagnoses    & 14567 \\  
    \hline
 Number of Procedures   & 3882  \\
    \hline
 Number of Medicine     & 4204  \\
    \hline
\end{tabular}
\end{center}

    \end{table}


\subsection{Testing and Evaluation}

There are three main things to test for evaluating the RS, performance,
accuracy scores and drug interactions tests. These three measures will
ensure that our system would recommend good medicine as fast as possible
whilst also ensuring the patient's safety. We will also compare our
system's safety with other RS using standard drug interactions metrics.

We will also split the dataset into training, testing, and validation sets
to calculate additional evaluation metrics such as the F1 score,
precision,Jaccard, Root-mean-square deviation, and Mean Absolute Errors. Doing so
will allow us to compare our system with existing systems that use the
MIMIC dataset. We could also package the system in an application and
present sample use cases to demonstrate our system. 




    \pagebreak

\section{Conclusion}

    \pagebreak
\bibliographystyle{IEEEtran}
\bibliography{/home/liamattard/Documents/Masters/bibtex/AImasters}

\end{document}
